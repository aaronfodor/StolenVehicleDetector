%----------------------------------------------------------------------------
\chapter{Summary}

In this work, I discussed the general task of automatic license plate recognition, then examined best practices for both object detection and optical character recognition. I created and trained deep learning-based algorithms to detect number plates and recognize texts on them. Then an entire ALPR pipeline was constructed. Then, I put together the different modules to form a complete vehicle identification system. In the end, I presented the fundamental concepts of my system's frontend and backend.

Further work would focus on many aspects of the proposed system. For example, the optical character recognition step could be improved by using the object detection approach instead of sequential processing. This way, a single algorithm could solve multiline text recognition. It is necessary to optimize the detector for inference speed: explicit predictor heads for each feature map on the computational graph would speed up the runtime. Another crucial step would be the usage of an anchor-free approach, like YOLOX\cite{YOLOX}, which would reduce the output tensor computation by orders of magnitude and simplify the pipeline. In my opinion, RetinaNet\cite{RetinaNet} loss is less robust than the YOLO approach, where a standalone objectness score is used to address the class imbalance problem. Therefore, I would like to try that variant and compare the results. The license plate and vehicle localization step could be optimized using a custom detector instead of the TensorFlow Object Detection API when these improvements are complete. Appropriate evaluation metrics are required to compare different models fairly. Therefore, implementing the COCO- or a custom metric is also needed to develop low-level detectors further. Using a network that recognizes the rotation angle of number plates could be a significant improvement because this would eliminate the need for a standalone plate rectification step. The OCR model could be much easier and faster, as it would not need to handle rotated characters anymore. Another development option would be to identify the number plate's country of origin and use a structural presumption. The current system is not constrained to license plates of specific countries. Still, if the territorial information could be extracted, that would make further processing easier (exactly how many characters are needed, whether a character is an ``O'' or a ``0'' based on the location on the plate).

A fair comparison of widespread ALPR systems and my solution is also essential. At the time of this work, I have not been granted access to the benchmark datasets (SSIG SegPlate and UFPR-ALPR).
