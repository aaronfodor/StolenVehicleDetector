\pagenumbering{roman}
\setcounter{page}{1}

\selecthungarian

%----------------------------------------------------------------------------
% Abstract in Hungarian
%----------------------------------------------------------------------------
\chapter*{Kivonat}\addcontentsline{toc}{chapter}{Kivonat}

Az elmúlt években a deep learning áttörést hozott különféle alkalmazási területekben, például a számítógépes látás és a természetes nyelvfeldolgozás terén. Ez a gépi tanulás jelenleg is aktívan kutatott területe, amely úgy tűnik, hatékony megoldásokat kínál számítógéppel nehezen algoritmizálható feladatokra. Ezzel párhuzamosan az okostelefonok robbanásszerű sebességgel fejlődnek. Az ARM CPU-k közelmúltbeli fejlődése, valamint a GPU-k megjelenése az asztali PC-khez hasonló számítási teljesítményt tesznek lehetővé.

A járműazonosítás gyakori feladat, amelyet gyakran az ALPR (automatikus rendszámtábla-felismerés) segítségével hajtanak végre. Ezt a technológiát használják az úthasználati engedélyek ellenőrzésére vagy lopott járművek keresésére. Az ilyen rendszerek már régóta léteznek, de fejlesztésük általában drága, működésükhöz egyedi hardver szükséges. A hagyományos ALPR rendszerek kézzel történő tulajdonság kinyerésre támaszkodnak, és független feldolgozási lépéseket tartalmaznak. Egy deep learning algoritmus képes magától megtanulni a szükséges tulajdonságok kinyerését; így robusztusabb megoldást nyújthat még változatos rendszámok felismerésében is.

Ebben a munkában egy valós idejű járműazonosító rendszert hozok létre. Először megvizsgálom az ALPR különböző megközelítéseit, majd javaslatot teszek egy valós idejű felhasználásra kialakított pipeline-ra. Ezután létrehozok deep learning algoritmusokat a teljes járműazonosítási folyamatának megvalósításához. Ezt követően bemutatom az elkészített Android alkalmazást, amely futtatja a kifejlesztett algoritmusokat. Ezután bemutatom a szerver alkalmazást is, melyen keresztül a felhasználók frissítése és a járművek bejelentése történik. Végül ismertetem az elkészített rendszer alkalmazhatóságát és korlátait.


\vfill
\selectenglish


%----------------------------------------------------------------------------
% Abstract in English
%----------------------------------------------------------------------------
\chapter*{Abstract}\addcontentsline{toc}{chapter}{Abstract}

In recent years, deep learning has shown breakthroughs in various applications, such as Computer Vision and Natural Language Processing. It is an actively researched area of machine learning that seems to provide effective solutions to tasks that have been difficult to solve with machines. Concurrently, smartphones are evolving at an explosive rate. The recent improvement of ARM CPUs and the advent of GPUs provide computing power comparable to desktop PCs.

Vehicle identification is a common task that is often accomplished through ALPR (Automatic License Plate Recognition). This technology is used to check for road usage permits or to look for stolen vehicles. Such systems have been around for a long time, but they are usually expensive to develop and require unique hardware to operate. Traditional ALPR relies on hand-crafted feature extraction and contains independent processing steps. Deep learning can learn feature extraction on its own; thus, it can provide a more robust solution even for recognizing various license plates.

In this work, I create a real-time vehicle identification system. First, I examine different ALPR approaches and propose a pipeline with real-time usability in mind. Then, I create and train deep learning algorithms for the complete vehicle identification process. Later, I present the Android application running the developed pipeline. After that, I also present the server application from which users can update themselves and report vehicles. Finally, I explain the applicability and the limitations of the prepared system.

\vfill
\selectthesislanguage

\newcounter{romanPage}
\setcounter{romanPage}{\value{page}}
\stepcounter{romanPage}